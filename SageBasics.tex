\documentclass{article}

\usepackage{amsmath,amssymb}
\usepackage{multicol}
\setlength\columnsep{1cm}
\usepackage{siunitx}
\usepackage[inline]{enumitem}
\usepackage{hyperref}
\usepackage{menukeys}
\renewmenumacro{\keys}[+]{shadowedangularkeys} %this package has confliction with TikZ "every node" styling
\usepackage{stackengine}
\def\stackalignment{l}

\usepackage{sagetex}


% for drawing graphs
\usepackage{tikz}
\tikzset{every picture/.style={thick}}
%\tikzset{every node/.style={draw, circle, inner sep = 2pt}}
\usetikzlibrary{arrows}

% for margins
\usepackage[left=0.5in, right=0.5in, top=0.8in, bottom=0.8in]{geometry}

% for font
\usepackage{euler}
\usepackage[OT1]{eulervm}
\renewcommand{\rmdefault}{pplx}

\setlength{\parindent}{0pt}  %no indenting


% for title
\title{Sage Basics}
\author{Jephian Lin\hspace{2cm}\today}
\date{\vspace{-1cm}}

\setlist[description]{font=\normalfont}
\newcommand{\heading}[1]{\medskip\textbf{#1}.}
\newenvironment{lwdes}{\begin{description}[leftmargin=1.7cm,align=left,labelwidth=1.5cm]}{\end{description}}

\begin{document}
\maketitle

\large

\begin{multicols}{2}
\heading{Python \& SageMath}  Python is a programming language.  It is easy to use and flexible for various purposes.  SageMath is an open-source mathematics software built on Python and stands for ``System for Algebra and Geometry Experimentation''.  It provides plenty of functions for mathematical computation and research.

\heading{How to install}  CoCalc is an online platform for Linux, \LaTeX{}, Python, Sage, R, etc.  It is free, but you may subscribe to get better performance.  Use CoCalc to experience the power of programming first.  
\begin{center}
\href{https://cocalc.com/}{CoCalc}
\hfil
\href{https://www.python.org/}{Python}
\hfil
\href{http://www.sagemath.org/}{SageMath}
\end{center}
Register an account of CoCalc, create a project, and create a Sage worksheet.  Then embrace the wonder of SageMath!

\heading{Your best friends}
\begin{lwdes}
\item[\stackanchor{\keys{shift}}{+\keys{enter}}] evaluate the cell
\item[\keys{tab}] autocomplete or show the possible completions
\item[\textit{object}\texttt{.}] press \texttt{tab} to see functions under \textit{object}
\item[\textit{func}\texttt{?}] evaluate this line to read the document of \textit{func}
\item[\textit{func}\texttt{??}] evaluate this line to read the source code of \textit{func} 
\item[Google] the answers are likely available online or in \href{https://doc.sagemath.org/html/en/reference/}{Sage Reference Manual}
\end{lwdes}

\heading{Assign a value}
\begin{lwdes}
\item[\texttt{a = 1}] set the value of \texttt{a} as \texttt{1}
\end{lwdes}

\heading{Print}
Run \texttt{print a} or \texttt{print(a)} to print the value of \texttt{a}.  Python 3 only accepts the second syntax.
\columnbreak

\heading{Data types}
\begin{lwdes}
\item[\texttt{int}] integers, such as \texttt{2}, \texttt{3}, \texttt{5}, $\ldots$\\
\texttt{Integer} is more common in Sage 
\item[\texttt{str}] strings, such as \texttt{"235"}
\item[\texttt{bool}] boolean values, namely, \texttt{True} and \texttt{False}
\item[\texttt{tuple}] tuples, such as \texttt{(2,3,5)}
\item[\texttt{list}] lists, such as \texttt{[2,3,5]}
\item[\texttt{dict}] dictionaries, such as \\\texttt{\{"two":2, "three":3, "five":5\}} \\defined by \texttt{\{\textit{key}:\textit{value}\}}
\item[\texttt{type(a)}] return the type of \texttt{a}
\end{lwdes}

\heading{Boolean tests}
\begin{lwdes}
\item[\texttt{in}] \texttt{1 in [2,3,5]} returns \texttt{False}
\item[\texttt{not in}] \texttt{1 not in [2,3,5]} returns \texttt{False}
\item[relation] \texttt{2==3} returns \texttt{False}\\
options: \texttt{>}, \texttt{>=}, \texttt{<}, \texttt{>=}, and \texttt{!=}\\
\texttt{!=} means not equal
\item[\texttt{isinstance}] check the type\\
\texttt{isinstance("235",str)} returns \texttt{True}
\end{lwdes}

\heading{Arithmetic operators}
\begin{lwdes}
\item[\texttt{+-*/}] addition, subtraction, and multiplication, division
\item[\texttt{**} or \texttt{\textasciicircum}] exponent, \texttt{**} for Python, \texttt{\textasciicircum} for Sage\\
\texttt{2\textasciicircum3} returns 8
\item[\texttt{\%}] modulus, \texttt{23\%4} returns \texttt{3}
\item[\texttt{\slash\slash}] floor division, \texttt{23\slash\slash4} returns \texttt{5} 
\end{lwdes}
\columnbreak

\heading{Layout}  Line breaks and indents are both sensitive in Python.  Conventionally, an indent is four spaces.  On CoCalc, I suggest go to ``Account'' and check the box of ``Spaces instead of tabs''.  If you put several commands in a line, then use semi-colons ``\texttt{;}'' to separate them.  Otherwise, semi-colons are optional.

\heading{The \texttt{if} statement}  The following code decides the letter grade of the input \texttt{score}.

\begin{verbatim}
score = 90;
if score >= 80 and score <= 100:
    print "A";
elif score >= 70 and score < 80:
    print "B";
elif score >= 60 and score < 70:
    print "C";
elif score >= 0 and score < 60:
    print "D";
else:
    print "Input score not valid";
\end{verbatim}

\heading{The \texttt{for} loop}  The following code prints the positive integers less than or equal to $100$ that is a multiple of $5$ or $7$.

\begin{verbatim}
for i in range(1,101):
    if i%5==0 or i%7==0:
        print i;
\end{verbatim}

You may use \texttt{generator} or \texttt{list} in a \texttt{for} loop.
\begin{lwdes}
\item[\texttt{range(b)}] the list $0,1,\ldots,b-1$.
\item[\texttt{range(a,b)}] the list $a,a+1,\ldots,b-1$
\item[\texttt{TreeIterator(n)}] the generator of trees on $n$ vertices, run the next line first\\
\makebox[\linewidth][r]{\texttt{from sage.graphs.trees import TreeIterator}}
\end{lwdes}

\heading{The \texttt{while} loop}  The following code is a primitive way to find the least common multiple of $5$ and $7$.

\begin{verbatim}
i=1;
while True:
    if i%5==0 and i%7==0:
        print i;
        break;
    else:
        i=i+1;
\end{verbatim}

Here \texttt{break} means to stop the loop.

\heading{Define a function}  The following function will return  the value of $\sum_{k=1}^n k^p$.

\begin{verbatim}
def power_sum(n,p,summand=False):
    total=0;
    for k in range(1,n+1):
        total += k^p;
        if summand:
            print k^p;
    return total;
\end{verbatim}

Thus, \verb|power_sum(10,1)| returns $55$.  The variable \texttt{summand} has a default value \texttt{False} so it is optional; when it is \texttt{True}, the function will print the summands.  For example, \verb|power_sum(10,2,True)| will print $1,4,9,\ldots,100$ and then return $385$.

% *arg and **kwarg

\heading{Call values}
\begin{lwdes}
\item[\texttt{f(a,b)}] return the value of the function \texttt{f} with given inputs \texttt{a} and \texttt{b}
\item[\texttt{L[k]}] return the value of the $k$-th element in the list \texttt{L}
\item[\texttt{D[k]}] return the value that corresponds to the the key \texttt{k} in the dictionary \texttt{D}
\end{lwdes}

\heading{Shorthand and string formatting}

\begin{obeylines}
\verb|[k^2 for k in range(6) if k%2==1]|
means\hfill \verb|[1,9,25]|
\verb|{k:k^2 for k in range(6) if k%2==1}|
means\hfill \verb|{1:1,3:9,5:25]|
\verb|n=5; print "%s+1=%s"%(n,n+1);|
prints\hfill \verb|5+1=6|
\verb|n=5; print "{0}+1={1}".format(n,n+1);|
prints\hfill \verb|5+1=6|
\end{obeylines}

\heading{Operations on a list}
Suppose \texttt{a=[0,1,2,3,4]}.
\begin{obeylines}
\verb|a[-2]|\hfill returns \verb|3|
\verb|a[1:-2]|\hfill returns \verb|[1,2]|
\verb|a[2:]+a[:2]|\hfill returns \verb|[2,3,4,0,1]|
\end{obeylines}
\columnbreak

\heading{Matrix}
To assign 
\[M=\begin{bmatrix}0&1&2\\3&4&5\end{bmatrix},\]
\begin{obeylines}
the following two lines do the same work.
\verb|M=matrix([[0,1,2],[3,4,5]]);|
\verb|M=matrix(2,range(6));|
\end{obeylines}

\medskip
Get information of a matrix \texttt{M} by the following ways.
\begin{lwdes}
\item[\texttt{M[i,j]}] the $i,j$-entry
\item[\texttt{M[[0,1],[1,2]]}] the submatrix induced on rows indexed by \texttt{[0,1]} and columns indexed by \texttt{[1,2]}.
\item[\texttt{M[[1],:]}] the row with index \texttt{1}
\item[\texttt{M[:,[2]]}] the column with index \texttt{2}
\end{lwdes}
\columnbreak

\heading{Graph}  To assign $G=K_{2,3}$, each of the following three lines achieve the task, but only the first line assigns the positions of the vertices.  (Try \texttt{G.show()} to see the differences.)
\begin{obeylines}
\verb|G=graphs.CompleteBipartiteGraph(2,3);|
\verb|G=Graph({0:[2,3,4],1:[2,3,4]});|
\verb|G=Graph("D]o");|
\end{obeylines}
Here \texttt{"D]o"} is the graph6 string of $K_{2,3}$.

\medskip
Use \href{http://pallini.di.uniroma1.it/}{\texttt{nauty}} to search graphs.  The following code prints the graph6 string for all connected graphs on $4$ vertices.
\begin{verbatim}
n=4;
for g in graphs.nauty_geng("%s -c"%n):
    print g.graph6_string();
\end{verbatim}

You may check the isomorphism by first giving them a ``standard'' labeling and then compare their strings.  The following code checks if \texttt{G} and \texttt{H} are isomorphic.
\begin{verbatim}
stgG=G.canonical_label().graph6_string();
stgH=H.canonical_label().graph6_string();
stgG==stgH;
\end{verbatim}

\end{multicols}

\end{document}

%%% SageTex example
Using Sage\TeX, one can use Sage to compute things and put them into
your \LaTeX{} document. For example, there are
$\sage{number_of_partitions(1269)}$ integer partitions of $1269$.
You don't need to compute the number yourself, or even cut and paste
it from somewhere.

Here's some Sage code:

\begin{sageblock}
    f(x) = exp(x) * sin(2*x)
\end{sageblock}

The second derivative of $f$ is

\[
  \frac{\mathrm{d}^{2}}{\mathrm{d}x^{2}} \sage{f(x)} =
  \sage{diff(f, x, 2)(x)}.
\]

Here's a plot of $f$ from $-1$ to $1$:

\sageplot{plot(f, -1, 1)}
 
%%% At the end, I didn't use SageTex...